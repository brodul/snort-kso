\documentclass[12pt]{article}
\usepackage[slovene]{babel}
\usepackage[utf8]{inputenc}
\usepackage{makeidx}
\usepackage{pifont}
\usepackage{listings}  
\usepackage{hyperref}  
\usepackage{graphicx}
\usepackage[margin=0.8in]{geometry}
\usepackage[xindy,toc,acronym]{glossaries}

\let\stdsection\section
\renewcommand\section{\newpage\stdsection}

\title{Sistemi za detekcijo napadov}
\author{Domen Kožar, Andraž Brodnik}
\begin{document}

\maketitle

\tableofcontents

\section{Povzetek}
Teoretični del zajema razlago sistemov za detekcijo napadov (IDS) in preprečevanje napadov (IPS) 
ter njihove podskupine. 
Osredotočili se bomo na sisteme, ki opazujejo promet na omrežnem vmesniku (NIDS/NIPS).
\\*
\\*
V praktičnem delu pa smo namestili in nastavili sistem Snort.

\section{Uvod}

Namen seminarske naloge se je seznaniti s sistemi za zaznavanje vdorov ter s sistemi za preprečevanje vdorov. 
Kako sestaviti osnovno politiko (policy) za tak sistem, kakšne napade lahko detektiramo, ter priporočljive obrambne mehanizme.
\\*
\\*
Dandanes se srecujemo z novicami o nepoblaščenih vdorih v informacijske sisteme. Taksni vdori lahko uničijo podjetje ali zasebnost uporabnikov,
kar pomeni, da je racionalno investirati nekaj tehničnih ur v postavitev sistema, ki bi lahko (ni pa nujno) taksen vdor preprečil ali pa zaznal poskus vdora.
To nam koristi, da vidimo na kaksen način je napadalec napadel naš sistem, ter kaj je storil. 
\\*
\\*
Kljub temu, da je naša varnostna politika v skladu z dobro prakso (menjava gesel, dvonivojska avtentikacija, požarni zidovi,
varne aplikacije, tuneliranje prometa, up-to-date strežniki), ne smemo biti preveč zadovoljni s sami sabo ter moramo postavit tudi sistem za detekcijo in/ali preprecevanje napadov.
\\*
\\*
Naj bralca opozorimo tudi na dejstvo, da `nepravilno' konfiguriran IDS ali IPS sistemi lahko globoko posežejo
tudi v zasebnost posameznika, kar ni v skladu z ustavo Republike Slovenije in drugimi pravnimi akti.
IDS in IPS sistemi se lahko uporabljajo tudi kot DPI (deep packet inspection) sistemi, kar pomeni, da ne gledamo
samo glav paketnih protokolov ampak tudi aplikacijski nivo (aplikacijski protokol oz. vsebino), 
zato nastavljajmo IDS in IPS sisteme odgovorno, podatke pa shranjujmo z največjo skrbjo.

\section{Teoretični del}

\subsection{Pregled tipov sistemov}
 
\subsubsection{IDS}

\begin{figure}[htb]
\begin{center}
\includegraphics[scale=0.8]{mac_flooding_attack.png}
\end{center}
\caption{MAC flooding napad}
\label{mac_flooding}
\end{figure}

Sistemi za detekcijo napadov (intrusion detection system), krajše IDS
so sistemi, katera naloga je analizirati podatke na omrezju ali sistemu samem
ter zaznati poskuse vdora ali pa vdor sam.

Naj omenimo, da so tej sistemi namenjeni ponudnikom storitev (podjetjem, inštitucijam, posameznikom)
v večini niso namenjeni omrežnim operaterjem, razen če želimo preprečevati napade na naše omrežne elemente. 
Ne moremo pa vsiljevati pravil za vse naše uporabnike. Včasih pa je bilo tega prometa za analizo preveč, a vendar
so se časi spremenili in to ni več glavna omejitev.

Delimo jih na dve glavni skupini:

\begin{itemize}
    \item - NIDS (network intrusion detection system)
    \item - HIDS (host intrusion detection system)
\end{itemize}

\subsubsection{NIDS}

Sistemi, katerim je glavni vir podatkov za analizo izključno omrežje se imenujejo NIDS sistemi (network intrusion detection system).
Kar pomeni, da opazujejo ves dohodni in izhodni promet, nato pa indentificira sumljive vzorce, ki bi lahko kazali na napad na omrezje ali nek sistem.

Bralcu bo po vsej verjetnosti poznan program WireShark ali pa tcpdump.
NIDS ponavadi delujejo podobno kot zgoraj omenjena programa. Program zajema vse paketke, ki jih vidi na omrežnem vmesniku,
nato jih premerja s pravili v svoji bazi, sumljive pakete ali niz paketov pa zabeleži ali pa si ustrezno napiše informacije o njih.

\paragraph{Komponente} % (fold)
\label{par:Komponente}

Sami NIDS sistemi so ponavadi razdeljeni na 2 dela:
\begin{itemize}
    \item - Senzor
    \item - Nadzorni del
\end{itemize}

Senzorji so programi, ki zajemajo patetke na dolocenem delu omrezja jih analizirajo nato pa posljejo nadzornem delu.

Taksna, delitev ima vec prednosti. Prva prednost je, da se dogodki/alarmi posiljajo naprej torej so reproducirani, tako je prakticno nemogoce za napadom pobrisati sledi.
Hkrati lahko zajemamo vec omrezjij, ki so na razlicnih lokacijah brez, da bi promet preosmerili na centralno lokacijo.
Oz. lahko dvignemo vec instanc (slovenski prevod primerek je neprimeren) senzorjev na enem samem strezniku, kar nam omogoca bolso uporabo racunskih virov na strezniku.
To nam seveda otezi samo nastavitev (konfiguracijo) sistema, kar pomeni, da je taksna postavitev smiselna za vecja podjetja.

% paragraph Komponente (end)

\paragraph{Detekcija napadov} % (fold)
\label{par:Detekcija napadov}

Kako poteka sama detakcija napadov? 
NIDS sistemi imajo vgrajeno bazo odtisov napadov. Ce je dogajanje na mrezi podobno, opisanem napadu v bazi
bo NIDS sistem to napisal v dnevnik ali pa si celo shranil napad (v pcap datoteki). 
NIDS sistemi so sumljicavi, glede velikosti, samega protokola in kolicine paketov.
NIDS sistemi imajo vgrajene tudi logicne avtomate, torej se zavedajo tudi paketov pred nekim paketom in po njem.

% paragraph Detekcija napadov (end)

\paragraph{Lokacija v omrezju} % (fold)
\label{par:Lokacija v omrezju}

NIDS sisteme lahko postavimo na vec delov v sistemu. Najbolj pogosto je za pozarnim zidom.
Z tem zajemamo vse pakete, ki jih pozarni zid ni zavrgel. To nam sicer ne, da celotnega vplogleda v vse napade.
Nam pa zmanjsa kolicino podatkov, ki jih moramo obdelati. Napadi , ki nam ne pridejo skozi pozarni zid naceloma niso nevarni.

NIDS senzorje imamo lahko razdeljene tudi na razlicnih fizicnih lokacijah ali logicnih lokacijah, ker ni mogoce vse realizirati z enim senzorjem ali pa si to ne zelimo (zahtevana je separacija).

% paragraph Lokacija v omrezju (end)

\paragraph{Omejitve NIDS} % (fold)
\label{par:Omejitve NIDS}

Pomembna mejitev je pogostost laznih alarmov.
Noben NIDS sistem ne more preprecit pojavljanje laznih alarmov.
V vecino NIDS sistemom je mogoce dodati tudi vzorce laznih alarmov. 
V bazo dodamo vzorec, ki nam sprozi alarm vendar NIDS sistem najde pravilo, da je to lazni alarm.


Sestavljanje TCP toka podatkov (stream)/Sestavljanje IP paketkov (zaradi defregmentacije)

Kot omenjeno kdaj analiziramo celotne TCP toke podatkov, da lahko zaznamo nevarnost. To pomeni, da moramo shranjevati paketke. 
Pri napadih mnogokrat ne zaklucimo toka podatkov, kar pomeni, da se more pri shranjevanju tokov podatkov NIDS obnasati zelo specificno. Podobne probleme imamo tudi pri IP paketih. 
Kot vemo imamo omejen polnilnik. Predstavljate si lahko koliko paketkov lahko shranimo na 10Gbit/s vmesniku.

\paragraph{Stvari, ki jih moramo premisliti} % (fold)
\label{par:Stvari, ki jih moramo premisliti}

% paragraph Stvari, ki jih moramo premisliti (end)

Pomemben je operaciski sistem. Ponavadi jo NIDS sistemi navoljo za vse sisteme,
tako Windoes NT kot Unix ter ostale. Neki sistemi lahko tecejo na vec razlicnih, tako ni nic nenavadnega
da imamo senzor na OpenBSD, managment pa na Windows NT sistemu.
Pomembno je da sistemski administratorji poznajo platformo na kateri tece NIDS ali del njega, saj je pomembno, da je ta ustrezno zascitena.

Podprti omrezni vmesniki
Pomembno je da se zavedamo, da niso podprti vsi vmrezni vmesniki. Vecina sistemov podpira samo ethernet vmesnike.

Alarmi
Pomembno je da se odlocimo kaksne nacine obvescanja se bomo posluzili. Ponavadi so detekcije (alarmi) razdeljeni v razlicne tipe ali nivoje.
Glede na katere dolocimo kanal obvescanja. 
Nekatere napade lahko zgolj napisemo v dnevnik, za nekatere posljemo e-posto skrbniku ali operatorju v nekaterih primerih pa posljemo SMS.

Pisanje dnevnikov in porocil
Vsi NIDS sistemi pisejo alarme v dnevnike, torej lahko za nazaj pogledamo, kaj se je dogajalo. Doloceni sistemi pa omogocaje tudi posiljanje avtomatsko generiranih porocil (npr. dnevnih).

Vzdrzevanje
Pomemben segment je tudi vzdrzevanje. Vprasanja glede tega so ali sem nam baza odtisov, sama posodablja ali pa jo moramo avtomatsko, kako se posodablja jedro. Moramo za posodobitve baze placevati. Koliko stane licenca?
Imamo vec alternativnih virov odtisov. Je okoli odtisov v bazi zbrana skupnost ali pa celo akademsko okolje.
Koliko fleksibilni so ti mehanizmi. Koliko pogosto moramo posadabljati. 
Stvar zavisi od nasih izkusenj in od izbranega sistema. Ni pa enolicnega odgovora za taksna vprasanja.

Izgled nadzornega dela
Pomemben je izgled nazornega dela. Ali je nadzorni del prijazen in ga hitro razumemo.
Nadzorni deli niso namenjeni konfiguraciji ampak predvsem prikazu alarmov, a lahko kjub temu omogocajo preproste nastavitve.
Vecina komercialnih sistemov nam ponuja graficne vmesnike. Realizirane za razlicne platforme ali pa kar spletni vmesnik.
Doloceni sistemi nam dajo na razpolago programerski vmesnik (API), nato pa vzamemo drug projekt za prikaz rezultatov.

Skalaribilnost
Skalaribilnost je pomemben aspekt, ce nacrtujemo sistem za omrezja, ki so velika ali pa se bodo razsirila. 
pri tej tocki, nas zanima ali NIDS sistem izkorisca vse procesorje na strezniku oz. kako to doseci ali lahko paktetke posiljamo naprej. Drugim racunalnikom in tako dosezemo neko kolektivno intelegenco.
Koliko paketkov lahko realno zajamemo in obdelamo, preden jih zacnemo spuscati (ne analizirati).


% paragraph Omejitve NIDS (end)

Tipicne sistemi v praksi so:
- Snort (odprtokodni sistem)
- Niksun NetDetector
- ISS RealSecurea

\subsubsection{HIDS}



\subsubsection{IPS}
\subsubsection{NIPS}
\subsubsection{HIPS}
\subsubsection{Primeri napadov}


\section{Praktični del}

\subsection{Nameščanje programske opreme Snort}
\subsection{Zmožnosti Snort programja}
\subsection{Spletni vmesnik za Snort}
\subsection{Praktični primeri napadov}

\section{Rezultati}

Bla bla.

\clearpage
\addcontentsline{toc}{section}{Viri}
\begin{thebibliography}{9}


\bibitem{eaptypes}
Snort Manual, \url{http://manual.snort.org/}, 20.12.2012

\end{thebibliography}
\end{document}



