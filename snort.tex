\documentclass[12pt]{article}
\usepackage[slovene]{babel}
\usepackage[utf8]{inputenc}
\usepackage{makeidx}
\usepackage{pifont}
\usepackage{listings}  
\usepackage{hyperref}  
\usepackage{graphicx}
\usepackage[margin=0.8in]{geometry}
\usepackage[xindy,toc,acronym]{glossaries}

\let\stdsection\section
\renewcommand\section{\newpage\stdsection}

\title{Sistemi za detekcijo napadov}
\author{Domen Kožar, Andraž Brodnik}
\begin{document}

\maketitle

\tableofcontents

\section{Povzetek}
Teoretični del zajema razlago sistemov za detekcijo napadov (IDS) in preprečevanje napadov (IPS) 
ter njihove podskupine. 
Osredotočili se bomo na sisteme, ki opazujejo promet na omrežnem vmesniku (NIDS/NIPS).
\\*
\\*
V praktičnem delu pa smo namestili in nastavili sistem Snort.

\section{Uvod}

Namen seminarske naloge se je seznaniti s sistemi za zaznavanje vdorov ter s sistemi za preprečevanje vdorov. 
Kako sestaviti osnovno politiko (policy) za tak sistem, kakšne napade lahko detektiramo, ter priporočljive obrambne mehanizme.
\\*
\\*
Dandanes se srecujemo z novicami o nepoblaščenih vdorih v informacijske sisteme. Taksni vdori lahko uničijo podjetje ali zasebnost uporabnikov,
kar pomeni, da je racionalno investirati nekaj tehničnih ur v postavitev sistema, ki bi lahko (ni pa nujno) taksen vdor preprečil ali pa zaznal poskus vdora.
To nam koristi, da vidimo na kaksen način je napadalec napadel naš sistem, ter kaj je storil. 
\\*
\\*
Kljub temu, da je naša varnostna politika v skladu z dobro prakso (menjava gesel, dvonivojska avtentikacija, požarni zidovi,
varne aplikacije, tuneliranje prometa, up-to-date strežniki), ne smemo biti preveč zadovoljni s sami sabo ter moramo postavit tudi sistem za detekcijo in/ali preprecevanje napadov.
\\*
\\*
Naj bralca opozorimo tudi na dejstvo, da `nepravilno' konfiguriran IDS ali IPS sistemi lahko globoko posežejo
tudi v zasebnost posameznika, kar ni v skladu z ustavo Republike Slovenije in drugimi pravnimi akti.
IDS in IPS sistemi se lahko uporabljajo tudi kot DPI (deep packet inspection) sistemi, kar pomeni, da ne gledamo
samo glav paketnih protokolov ampak tudi aplikacijski nivo (aplikacijski protokol oz. vsebino), 
zato nastavljajmo IDS in IPS sisteme odgovorno, podatke pa shranjujmo z največjo skrbjo.

\section{Teoretični del}

\subsection{Pregled tipov sistemov}
 
\subsubsection{IDS}

\begin{figure}[htb]
\begin{center}
\includegraphics[scale=0.8]{mac_flooding_attack.png}
\end{center}
\caption{MAC flooding napad}
\label{mac_flooding}
\end{figure}

Sistemi za detekcijo napadov (intrusion detection system), krajše IDS
so sistemi, katera naloga je analizirati podatke na omrezju ali sistemu samem
ter zaznati poskuse vdora ali pa vdor sam.

Naj omenimo, da so tej sistemi namenjeni ponudnikom storitev (podjetjem, inštitucijam, posameznikom)
v večini niso namenjeni omrežnim operaterjem, razen če želimo preprečevati napade na naše omrežne elemente. 
Ne moremo pa vsiljevati pravil za vse naše uporabnike. Včasih pa je bilo tega prometa za analizo preveč, a vendar
so se časi spremenili in to ni več glavna omejitev.

Delimo jih na dve glavni skupini:

\begin{itemize}
    \item - NIDS (network intrusion detection system)
    \item - HIDS (host intrusion detection system)
\end{itemize}

\subsubsection{NIDS}

Sistemi, katerim je glavni vir podatkov za analizo izključno omrežje se imenujejo NIDS sistemi (network intrusion detection system).
Kar pomeni, da opazujejo ves dohodni in izhodni promet, nato pa indentificira sumljive vzorce, ki bi lahko kazali na napad na omrezje ali nek sistem.

Bralcu bo po vsej verjetnosti poznan program WireShark ali pa tcpdump.
NIDS ponavadi delujejo podobno kot zgoraj omenjena programa. Program zajema vse paketke, ki jih vidi na omrežnem vmesniku,
nato jih premerja s pravili v svoji bazi, sumljive pakete ali niz paketov pa zabeleži ali pa si ustrezno napiše informacije o njih.

\paragraph{Komponente} % (fold)
\label{par:Komponente}

Sami NIDS sistemi so ponavadi razdeljeni na 2 dela:
\begin{itemize}
    \item - Senzor
    \item - Nadzorni del
\end{itemize}

Senzorji so programi, ki zajemajo patetke na dolocenem delu omrezja jih analizirajo nato pa posljejo nadzornem delu.

Taksna, delitev ima vec prednosti. Prva prednost je, da se dogodki/alarmi posiljajo naprej torej so reproducirani, tako je prakticno nemogoce za napadom pobrisati sledi.
Hkrati lahko zajemamo vec omrezjij, ki so na razlicnih lokacijah brez, da bi promet preosmerili na centralno lokacijo.
Oz. lahko dvignemo vec instanc (slovenski prevod primerek je neprimeren) senzorjev na enem samem strezniku, kar nam omogoca bolso uporabo racunskih virov na strezniku.
To nam seveda otezi samo nastavitev (konfiguracijo) sistema, kar pomeni, da je taksna postavitev smiselna za vecja podjetja.

% paragraph Komponente (end)

\paragraph{Detekcija napadov} % (fold)
\label{par:Detekcija napadov}

Kako poteka sama detakcija napadov? 
NIDS sistemi imajo vgrajeno bazo odtisov napadov. Ce je dogajanje na mrezi podobno, opisanem napadu v bazi
bo NIDS sistem to napisal v dnevnik ali pa si celo shranil napad (v pcap datoteki). 
NIDS sistemi so sumljicavi, glede velikosti, samega protokola in kolicine paketov.
NIDS sistemi imajo vgrajene tudi logicne avtomate, torej se zavedajo tudi paketov pred nekim paketom in po njem.

% paragraph Detekcija napadov (end)

\paragraph{Lokacija v omrezju} % (fold)
\label{par:Lokacija v omrezju}



% paragraph Lokacija v omrezju (end)

Tipicne sistemi v praksi so:
- Snort (odprtokodni sistem)
- Niksun NetDetector

\subsubsection{HIDS}



\subsubsection{IPS}
\subsubsection{NIPS}
\subsubsection{HIPS}
\subsubsection{Primeri napadov}


\section{Praktični del}

\subsection{Nameščanje programske opreme Snort}
\subsection{Zmožnosti Snort programja}
\subsection{Spletni vmesnik za Snort}
\subsection{Praktični primeri napadov}

\section{Rezultati}

Bla bla.

\clearpage
\addcontentsline{toc}{section}{Viri}
\begin{thebibliography}{9}


\bibitem{eaptypes}
Snort Manual, \url{http://manual.snort.org/}, 20.12.2012

\end{thebibliography}
\end{document}



